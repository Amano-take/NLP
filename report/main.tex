\documentclass[12pt]{article}
\usepackage[utf8]{inputenc}
\usepackage{amsmath}
\usepackage{graphicx}
\usepackage{geometry}
\geometry{a4paper, left=30mm, right=30mm, top=30mm, bottom=30mm}

\title{二文字入力方式の実装と評価}
\author{天野 岳洋 学生番号 6930-36-6945}
\date{\today}

\begin{document}

\maketitle

\begin{abstract}
本研究では、統計的な二文字入力方式の実装を行い、n-gramモデル、RNN、KyTeaベースの3つのアプローチを比較する。
これらの手法の性能を評価するためにtext8というデータセットを準備し、トレーニング部分でモデルを訓練し、テスト部分で評価を行いました。結果として、各モデルの精度と有効性を報告する。
\end{abstract}

\section{はじめに}
近年、入力方式の効率と精度は、モバイルテキストからプロフェッショナルなライティングツールまで、ユーザーエクスペリエンスを向上させるために重要な要素となっている。
そこで本研究では、二文字入力方式の実装を通じて、入力効率の向上を目指しました。具体的には、n-gramモデル、リカレントニューラルネットワーク(RNN)、およびKyTeaを用いた3つのアプローチを検討しました。


\section{方法}
本研究では、以下の3つのアプローチを用いて統計的な二文字入力方式を実装しました。

\subsection{n-gramモデル}
n-gramモデルを使用して、先頭2文字を基に次に続く文字を予測しました。n-gramのサイズやスムージング技法などの詳細は、補完してください。

\subsection{リカレントニューラルネットワーク(RNN)}
RNNを使用して、文字列の連続性を考慮しながら次の文字を予測しました。RNNのアーキテクチャやトレーニングパラメータなどの詳細は、補完してください。

\subsection{KyTeaベースのセグメンテーション}
KyTeaを使用してテキストを分割し、二文字入力方式の予測精度を向上させました。KyTeaの設定や使用方法などの詳細は、補完してください。

\section{言語リソース(データセット)}
本研究では、データセットとして適当なテキストコーパスを使用しました。データセットの準備方法や内容については以下の通りです。

\begin{table}[h]
\centering
\begin{tabular}{|c|c|}
\hline
項目 & 詳細 \\
\hline
データセット名 & \_\_\_\_\_ \\
トレーニングデータのサイズ & \_\_\_\_\_ \\
テストデータのサイズ & \_\_\_\_\_ \\
言語 & 日本語 \\
ソース & \_\_\_\_\_ \\
\hline
\end{tabular}
\caption{データセットの仕様}
\end{table}

\section{実験評価}
各モデルの性能を評価するために、トレーニングデータとテストデータを使用して実験を行いました。評価指標としては精度を用いました。各手法の精度は以下の通りです。

\subsection{n-gramモデルの評価}
n-gramモデルの評価結果について補完してください。

\subsection{RNNの評価}
RNNの評価結果について補完してください。

\subsection{KyTeaベースの評価}
KyTeaベースの評価結果について補完してください。

\section{結論}
本研究では、統計的な二文字入力方式の3つのアプローチを比較しました。それぞれの手法において異なる利点と欠点が見られましたが、全体的には\_\_\_\_\_が最も高い精度を示しました。今後の研究では、さらに大規模なデータセットを使用してモデルの改善を図るとともに、ユーザーインターフェースの最適化も検討していきます。

\end{document}
